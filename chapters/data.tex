\section{The Impact of Language and Improvisation}
\label{sec:data}
Training and validating models for research is usually done on the same dataset. This leads to solid proof-of-concept models, but can lead to models that do not generalize well to real world scenarios. There is also another aspect of robustness of FER models against speech we have not touched on yet, which is the language dimension. We want to provide a platform to test our models against a different language, to see if this modality impacts performance. Another point to note is that the datasets we used are scripted and acted. Real world situations are not as rigid and more closely follow speech acts.

In this section we will present our approach to build a small German database along the lines of RAVDESS, and collect videos of speech-acts that more closely resemble those of real world scenarios to later validate our models on them.

\subsection{German RAVDESS}
\label{sec:german}

The lexical content of RAVDESS is kept simple. The actors speak two easy to pronounce sentences in eight emotions. We replicate this by switching the English sentences for German counterparts. We chose "Die Katze sitzt auf dem Sofa" and "Die Kinder spielen im Garten", two sentences with eight syllables and commonly used words. 

Recordings were done using the TAWNY recording tool \cite{tawny2021}. The actors were given a general description of the task before starting their session. Before each individual recording the actors were given the target sentence and emotion, alongside an example from the RAVDESS dataset.

14 actors participated, 8 female and 6 male. We checked the videos for visual quality and sufficient emotional perception. After quality checking, we end up with 141 recordings from 11 actors.

\subsection{Real World Speech Acts}

To collect statements that fit in real world scenarios, we put the actors in situations where they had to react with their own words. Four scenarios were built to illicit certain responses:

\begin{enumerate}
    \item \textbf{Promise} "Together with one of your coworkers, you have just finished your lunch at a local restaurant. When it is time to pay, you realise that you forgot your wallet at the office. Knowing how stingy your colleague can be, try to ask him if you could borrow some money and affirm that you will give it back once you’re back at work."
    \item \textbf{Order} "You are the team lead of the sales team. You currently have the weekly meeting with your team members. Given that the yearly report is about to be due, it’s a very important week ahead. Remind Thomas that he is in charge of creating the final draft of the report, and tell him that it is due until next Friday."
    \item \textbf{Apology} "You’ve met with your best friend for coffee at a local cafe. You offered to fetch coffee for both of you and bring them to the table. After you’re back and have setteled, your friend realised that you’ve given the wrong order, and brought back a Latte Macchiato instead of a regular Americano. Apologize for the mixup."
    \item \textbf{Assessment} "You are the team lead of the sales team. After a long and busy week, you meet with your team members for a debrief. Thomas has done a great job on the final draft of the yearly report, which he handed in on time. Compliment him on his great work."
\end{enumerate}

The recordings were collected with the TAWNY recording tool \cite{tawny2021}. The actors were confronted with the situation, and had time to make up a response before being recorded. Five people participated, one male and four female.